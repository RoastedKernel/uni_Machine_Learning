\documentclass{article}

%\usepackage{natbib}
\usepackage{harvard}

%\title{My first document}
%\date{2013-09-01}
%\author{John Doe}


\begin{document}
%\maketitle
	\newpage
\section{Human Factors}
\subsection{Migration and Environment }
The web-based application will be closed and replaced for a new mobile system in a matter of one month. Such transition will greatly impact the clients that used the software for 10 years on the web based system. They got accustomed to a particular flow of software operation. Changing that flow and replacing it with a mobile or phone based flow is a transition that will debit the old client base. The fast transition alone is a challenge changing the environment also is another challenge. Majority of the users would have used the web based system using a relatively large screen with a keyboard and a mouse but after a month they will have to replace the keyboard and the mouse with a small touch screen relative a big computer screen. Both the Short transition time and the change in environment represent human factor issues. To mitigate such problems. The transition period could be extended to allow time for the old client base to properly get accustomed to the new app environment. Maintaining as much as possible of the old web based user interface will greatly help the transition to try to maintain the same flow as the web based system. 


\subsection{Yearly Access}
The users must access the system yearly to keep their account live. Such requirement can be considered as  negative friction. Requiring users the access the system yearly to maintain their data is a hindrance to the user's overall experience. An email can be sent to the user's instead of deleting their data without any warning hence it can be tiresome and maybe difficult for some users to remember to access the system if they are not using it repeatedly . 



\subsection{Two-Factor Authentication}
Requiring the users to use Two-Factor authentication also can be considered as negative friction. Two-factor authentication is still a new concept especially for the user base that has been using the web based system will find using such method difficult and unfriendly. A simple user name and password will suffice and two-Factor authentication can still be used in the instance of a user forgot the password or the username. In that case the a positive user's experience can be maximized while not compromising much on the security of their accounts. 

\newpage

\section{Ethical Factors}
\subsection{Daily Login Requirement}
Forcing the users to log in the app daily for security purposes is an an usual way to ensure the safety of their data in the event of some other entity gained access to the user's phone. One can argue that having a user name and a good password is sufficient enough to satisfy that requirement. But forcing the users to log in daily can be viewed as a method to maximize the revenue from the advertising hence using the app only when the user is traveling wont generate as much revenue as users that have to log in daily. Using security as a reason to force users to log in into the app for the purpose of maximizing the app revenue is a major ethical issue. As mentioned before a user name and a password should suffice the typical security concerns. as for maximizing app revenue the developers can focus on adding new features that will entice the users to use the app more often. 




\subsection{Migration Incentives and Data Rights}
Only the people on the mobile system will have access to their records. This is a very obvious and blatant move to force users to migrate to the app. The fact that there are no methods for the users to gain access to their data without using the app its as if the app developers are using the users own data as leverage dare say even hostage "to gain access to your data use the app". Using user's own data as leverage to passive aggressively make sure they use the app is unethical but that action itself gives the impression that developers OWN the users data as if its their own intellectual property that they can save and hold in their possession. Even if that wasn't the developers intentions and they are other technological limitation that forced them to such decision following saying encapsulates the developers choices "the road to failure is paved with good intentions". Assuming the developers will extend the transition period. A campaign can be launched to highlight the benefits of using the mobile based system. For example having live updates of the User' current flight details if the flight will be delayed. What is the current weather in home and wanted destination and more. All that at the tip of the user's fingers instead of accessing a website. Seeing such benefits at the transition period the users will gravitate to whatever provides the best features. 



\subsection{Data Collection}
Since advertising is going to be used to offset some of the costs a rage important questions arises. Will the users data be used for advertisement purpose? and if yes to what degree? can the users  be identified through what ever data the company shares?. All are important questions and the answers will have an huge effect on the user's data rights and privacy. Navigating such questions is a daunting task. Developers tends to treat every problem as a challenge. Such methodology can be taken to the extreme. As an example researchers have used sophisticated modeling technique to uncover the identity of graffiti artist called "Banksy" using only publicly available data\cite{Hauge2016}. Such methods can be of a great use in case trying to locate a serial killer. The researcher used an unidentified graffiti artist as a "challenge". the researchers abstracted the consequences or didn't take the repercussion of using such methods to win the "challenge" into consideration. is it ethical to release such information about a person that didn't explicitly agree to such experiment?, by releasing such information can it be used to harm the identified individual? and more. They are numerous example where researcher or individuals that work in the STEM field as are developers that concentrate on the "challenge" without careful consideration on the effects of their actions or choices from an ethical stand point. luckily frameworks, laws and standards are being developed such as \cite{Zook2017} that can help such individuals to navigate answering such questions.     

\bibliography{MyCollection} 
\bibliographystyle{agsm}

\end{document}