\documentclass{article}

%\usepackage{natbib}
\usepackage{harvard}

%\title{My first document}
%\date{2013-09-01}
%\author{John Doe}


\begin{document}
%\maketitle
	\newpage
\section{Informed Consent}

Informed consent by definition should be informed. the participants should be provided with all the necessary information and explicit details on how their data will be used and stored. To the extend that the participants should be encouraged to ask question, enquire and ask for better clarification in case anything isn't clear. once the participant has full understanding on what exactly is being shared and how only then the participant can make an informed decision. Consent must be given voluntary it cant be subdued or coerced from the participant. an option should be available to opt in or out from the data collection process. the details on what is exactly is the participant sharing should be specific and written in a language the participant can understand and comprehend easily. 


\section{Utilitarian Argument}
In the provided scenario. The medical practice is confronted with a problem that has been consistent that a number of patients have been sending complaints. The issue is some patients have notified the practice that doctors aren't spending enough time with patients. The owners of the medical practice have decided that using the data available to analyze if the data can reveal any pattern that can explain the reason for the issue. using the utilitarian lens as "good consequence = good action" the action in this scenario is analyzing patients data to identify the reason for the problem, assuming the analysis resulted in solution and the practice did manage to rectify the issue it is considered as good action that benefited the practice and future patients. in hindsight the notion of basing your action on whatever results in the maximum benefit for the largest amount of people seems an argument that no one would dispute at the first glance. Utilitarian ethics doesn't provide a concrete definition of what is considered good. Goodness is a very subjective term. take for example the use of enhanced interrogation techniques on prisoners. does torturing the prisoners for the sake of extracting information that could save innocent people's lives is good? is it ethical?. Following the Utilitarian argument in this hypothetical scenario. As long as the  total lives saved vastly overcome total pain caused it is morally justified in-fact some researchers have argued that is morally required. in both provided scenario and the hypothetical scenario have resulted in a good outcome for the patients and lives saved respectively. even though both scenario resulted in a net good outcome one had risks of leaking or breaking privacy of the patients and other caused pain and suffering. Which illustrates nuances of the nature of “good”.

\newpage

\section{Ethical Factors}
\subsection{Daily Login Requirement}
Forcing the users to log in the app daily for security purposes is an an usual way to ensure the safety of their data in the event of some other entity gained access to the user's phone. One can argue that having a user name and a good password is sufficient enough to satisfy that requirement. But forcing the users to log in daily can be viewed as a method to maximize the revenue from the advertising hence using the app only when the user is traveling wont generate as much revenue as users that have to log in daily. Using security as a reason to force users to log in into the app for the purpose of maximizing the app revenue is a major ethical issue. As mentioned before a user name and a password should suffice the typical security concerns. as for maximizing app revenue the developers can focus on adding new features that will entice the users to use the app more often. 




\subsection{Migration Incentives and Data Rights}
Only the people on the mobile system will have access to their records. This is a very obvious and blatant move to force users to migrate to the app. The fact that there are no methods for the users to gain access to their data without using the app its as if the app developers are using the users own data as leverage dare say even hostage "to gain access to your data use the app". Using user's own data as leverage to passive aggressively make sure they use the app is unethical but that action itself gives the impression that developers OWN the users data as if its their own intellectual property that they can save and hold in their possession. Even if that wasn't the developers intentions and they are other technological limitation that forced them to such decision following saying encapsulates the developers choices "the road to failure is paved with good intentions". Assuming the developers will extend the transition period. A campaign can be launched to highlight the benefits of using the mobile based system. For example having live updates of the User' current flight details if the flight will be delayed. What is the current weather in home and wanted destination and more. All that at the tip of the user's fingers instead of accessing a website. Seeing such benefits at the transition period the users will gravitate to whatever provides the best features. 



\subsection{Data Collection}
Since advertising is going to be used to offset some of the costs a rage important questions arises. Will the users data be used for advertisement purpose? and if yes to what degree? can the users  be identified through what ever data the company shares?. All are important questions and the answers will have an huge effect on the user's data rights and privacy. Navigating such questions is a daunting task. Developers tends to treat every problem as a challenge. Such methodology can be taken to the extreme. As an example researchers have used sophisticated modelling technique to uncover the identity of graffiti artist called "Banksy" using only publicly available data\cite{Hauge2016}. Such methods can be of a great use in case trying to locate a serial killer. The researcher used an unidentified graffiti artist as a "challenge". the researchers abstracted the consequences or didn't take the repercussion of using such methods to win the "challenge" into consideration. is it ethical to release such information about a person that didn't explicitly agree to such experiment?, by releasing such information can it be used to harm the identified individual? and more. They are numerous example where researcher or individuals that work in the STEM field as are developers that concentrate on the "challenge" without careful consideration on the effects of their actions or choices from an ethical stand point. luckily frameworks, laws and standards are being developed such as \cite{Zook2017} that can help such individuals to navigate answering such questions.     

\bibliography{MyCollection} 
\bibliographystyle{agsm}

\end{document}