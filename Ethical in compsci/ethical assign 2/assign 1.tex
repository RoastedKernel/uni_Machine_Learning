\documentclass{article}

%\usepackage{natbib}
\usepackage{harvard}

%\title{My first document}
%\date{2013-09-01}
%\author{John Doe}


\begin{document}
%\maketitle
	\newpage
\section{Informed Consent}

Informed consent by definition should be informed. the participants should be provided with all the necessary information and explicit details on how their data will be used and stored. To the extend that the participants should be encouraged to ask question, enquire and ask for better clarification in case anything isn't clear. once the participant has full understanding on what exactly is being shared and how only then the participant can make an informed decision. Consent must be given voluntary it cant be subdued or coerced from the participant. an option should be available to opt in or out from the data collection process. the details on what is exactly is the participant sharing should be specific and written in a language the participant can understand and comprehend easily. 


\section{Utilitarian Argument}
In the provided scenario. The medical practice is confronted with a problem that has been consistent that a number of patients have been sending complaints. The issue is some patients have notified the practice that doctors aren't spending enough time with patients. The owners of the medical practice have decided that using the data available to analyze if the data can reveal any pattern that can explain the reason for the issue. using the utilitarian lens as "good consequence = good action" the action in this scenario is analyzing patients data to identify the reason for the problem, assuming the analysis resulted in solution and the practice did manage to rectify the issue it is considered as good action that benefited the practice and future patients. in hindsight the notion of basing your action on whatever results in the maximum benefit for the largest amount of people seems an argument that no one would dispute at the first glance. Utilitarian ethics doesn't provide a concrete definition of what is considered good. Goodness is a very subjective term. take for example the use of enhanced interrogation techniques on prisoners. does torturing the prisoners for the sake of extracting information that could save innocent people's lives is good? is it ethical?. Following the Utilitarian argument in this hypothetical scenario. As long as the  total lives saved vastly overcome total pain caused it is morally justified in-fact some researchers have argued that is morally required. in both provided scenario and the hypothetical scenario have resulted in a good outcome for the patients and lives saved respectively. even though both scenario resulted in a net good outcome one had risks of leaking or breaking privacy of the patients and other caused pain and suffering. Which illustrates nuances of the nature of “good”.  Utilitarianism can also be applied to decide on viewing groups. the report produced in such scenario holds very sensitives information about the patients and the practice itself. of course the patients information can be anonymized as an effort to protect the patients privacy, but if the such report or data got into the hands of individuals that can benefit from such information even anonymization can be circumvented. With the emergence of big data and the vast information available due to social media and the higher accessibility to scarping tools that can scrap, collect vast amount of data that can be used to de-anonymize the patients records. taking into consideration the risks associated working with big data and medical records. Under the Utilitarian approach, the report should be accessible to the individualize that can be reading and comprehending the findings of the report can reap the most benefits to the patients. In this scenario and individual that is capable to understand the complexities of medical practices and able to articulate distilled findings and recommendation that the practice owner can act upon. 

%\newpage

\section{•}  



\newpage
\bibliography{MyCollection} 
\bibliographystyle{agsm}

\end{document}